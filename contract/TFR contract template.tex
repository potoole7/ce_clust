% %%
%% MMath Project Specification
%%
\documentclass[11pt]{article}
\thispagestyle{empty}

\setlength{\voffset}{-0.75in}
\setlength{\headsep}{5pt}

% %% Info for Top and Bottom Matter

% \newcommand{\student}{Student name}
\newcommand{\student}{Patrick O'Toole}
\newcommand{\ProjectTitle}{Developing clustering methods for conditional extreme value models}
% \newcommand{\supervisor}{Supervisor name(s)}
\newcommand{\supervisor}{Christian Rohrbeck}
%\newcommand{\checker}{P. Moerters}
\newcommand{\CoD}{Thomas Burnett}
\newcommand{\deadline}{6th September 2024}

\begin{document}

%% Top Matter

\begin{center}
\bfseries{\Large MA50248: SAMBa MRes Project}\\[6pt]
% \large\ProjectTitle
\ProjectTitle
\end{center}
% \medskip
% \smallskip
\begin{center}
\begin{tabular}{ll}
Student:&\student\\
Supervisor/1st examiner:&\supervisor\\
%Checker/2nd examiner:&\checker\\
Second reader:&\CoD\\
Submission deadline:&\deadline
\end{tabular}
\end{center}

%% Project Description

% conditional extremes -> Clustering conditional extremess

{\bf Project Description:}
% Fill out a rough project description and use a variant of the breakdown below for marking purposes. The latter can be changed according to the project although 10/100 MUST be for the oral presentation. 
Conditional extreme value models have proven useful for analysing the joint tail behaviour of random vectors. 
While an extensive amount of work to estimate conditional extremes models exists in multivariate and spatial applications, the prospect of clustering for models of this type has not yet been explored. 
This project will review existing methods in the area of conditional extremes models, and develop and research ideas on how some of these models can be embedded into a clustering framework. 
It will also involve the review of existing state-of-the-art clustering methods within extreme value analysis. 

%% Project Guidelines

% {\bf Project Guidelines:} {\it This is just an example.} Every two weeks the student will give a short presentation on a section of the reading during that period (this is informal and no credit will be given). The student will conclude by writing a coherent and original exposition of the large majority of topics covered. The report should be approximately 40 pages in length.

{\bf Project Guidelines:} Every week the student will meet with the supervisor to give a short summary of the reading and work conducted over the past week.
The student will conclude by writing a coherent and original exposition of the large majority of topics covered. The report should be approximately 40 pages in length.

% \medskip

%% Mark Breakdown

\begin{center}
\textbf{Assessment}\\[6pt]
\begin{tabular}{lr}
Personal Initiative: &10\\
Use of the Literature: &15\\
Presentation of Report: &15\\
%Bi-weekly presentations: &10\\
Contents of Report (depth, breadth, accuracy): &50 \\
Oral Presentation (Must be on 4th or 11th October): & 10\\
\cline{2-2}
TOTAL&100
\end{tabular}
\end{center}

%% Bottom Matter

\begin{center}
\textbf{Signatures}\\[6pt]
\begin{tabular}[t]{*{4}{p{1.2in}}}
Student&Supervisor&Second Reader\\[1in]
\student&\supervisor&\CoD
\end{tabular}
\end{center}

\end{document}

